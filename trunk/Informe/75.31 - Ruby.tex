\documentclass{article}

%% PAQUETES

% Paquetes generales
\usepackage[margin=2cm, paperwidth=210mm, paperheight=297mm]{geometry}
\usepackage[spanish]{babel}
\usepackage[utf8]{inputenc}
\usepackage{gensymb}

% Paquetes para estilos
\usepackage{textcomp}
\usepackage{setspace}
\usepackage{colortbl}
\usepackage{color}
\usepackage{color}
\usepackage{upquote}
\usepackage{xcolor}
\usepackage{listings}
\usepackage{caption}
\usepackage[T1]{fontenc}
\usepackage[scaled]{beramono}

% Paquetes extras
\usepackage{amssymb}
\usepackage{float}
\usepackage{graphicx}

%% Fin PAQUETES


% Definición de preferencias para la impresión de código fuente.
%% Colores
\definecolor{gray99}{gray}{.99}
\definecolor{gray95}{gray}{.95}
\definecolor{gray75}{gray}{.75}
\definecolor{gray50}{gray}{.50}
\definecolor{keywords_blue}{rgb}{0.13,0.13,1}
\definecolor{comments_green}{rgb}{0,0.5,0}
\definecolor{strings_red}{rgb}{0.9,0,0}

%% Caja de código
\DeclareCaptionFont{white}{\color{white}}
\DeclareCaptionFont{style_labelfont}{\color{black}\textbf}
\DeclareCaptionFont{style_textfont}{\it\color{black}}
\DeclareCaptionFormat{listing}{\colorbox{gray95}{\parbox{16.78cm}{#1#2#3}}}
\captionsetup[lstlisting]{format=listing,labelfont=style_labelfont,textfont=style_textfont}

\lstset{
	aboveskip = {1.5\baselineskip},
	backgroundcolor = \color{gray99},
	basicstyle = \ttfamily\footnotesize,
	breakatwhitespace = true,   
	breaklines = true,
	captionpos = t,
	columns = fixed,
	commentstyle = \color{comments_green},
	escapeinside = {\%*}{*)}, 
	extendedchars = true,
	frame = lines,
	keywordstyle = \color{keywords_blue}\bfseries,
	language = Octave,                       
	numbers = left,
	numbersep = 5pt,
	numberstyle = \tiny\ttfamily\color{gray50},
	prebreak = \raisebox{0ex}[0ex][0ex]{\ensuremath{\hookleftarrow}},
	rulecolor = \color{gray75},
	showspaces = false,
	showstringspaces = false, 
	showtabs = false,
	stepnumber = 1,
	stringstyle = \color{strings_red},                                    
	tabsize = 2,
	title = \null, % Default value: title=\lstname
	upquote = true,                  
}

%% FIGURAS
\captionsetup[figure]{labelfont=bf,textfont=it}

% COMANDOS

%% Titulo de las cajas de código
\renewcommand{\lstlistingname}{Código}
%% Titulo de las figuras
\renewcommand{\figurename}{Figura}
%% Referencia a los códigos
\newcommand{\refcode}[1]{\textit{Código \ref{#1}}}
%% Referencia a las imagenes
\newcommand{\refimage}[1]{\textit{Imagen \ref{#1}}}



\begin{document}

% Inserción del título, autores y fecha.
\title{\huge 75.31 Teoría de Lenguage \\ 
	  \Huge El lenguaje de programación Ruby \\
	  \bigskip \Large 28 de mayo de 2012 \\
	  \bigskip\bigskip \large\textit{Diaz, Federico (83568)\\Rossi, Federico Martín (92086)}}
\date{}
\maketitle




% INTRODUCCIÓN
\section{Introducción: \textit{Enfoque de la presentación}}

La presentación, como así también el presente informe, tienen como fin mostrar a los oyentes este veterano, y a la vez moderno lenguaje de programación desde un punto de vista diferente al que generalmente se acostumbra. Es nuestro objetivo el dar a conocer las fortalezas y debilidades de éste fijando la perspectiva de la presentación en las aplicaciones cotidiantas del lenguaje en el desarrollo de proyectos profesionales. Es decir, sin quitar importancia a tópicos importantes como la sintaxis, nos focalizaremos en detallar aquellas caracteristicas que hacen de Ruby un lenguaje interesante y ávido de nuevos conceptos.




% Un poco de historia
\section{Un poco de historia}

Ruby fue creado por \textit{Yukihiro ``Matz'' Matsumoto}. Matz comenzó a trabajar en Ruby en 1993, cuando tenia 28 años, mientras trabajaba para una empresa Open source (netlab.jp). Ya por ese entonces era muy conocido en Japón por tener un alto perfil de evangelista dentro de la comunidad Open source y por trabajar en varios productos Open source. Ruby es el primer software japones altamente conocido fuera de Japón, el cual fue presentado en 1995.

\begin{quotation}
\em``En 1993, yo estaba hablando con un colega acerca de lenguajes de scripting. Estaba muy impresionado por su poder y sus posibilidades. Sentía que el scripting era el camino a seguir.
	\par
	Me pareció que la programación orientada a objetos (POO) era muy adecuada para secuencias de comandos también (como viejo fan de la POO). Luego miré alrededor de la red. Me encontré con que Perl 5, que no se había lanzado todavía, implementaría las características de OO, pero  finalmente no resultó lo que esperába. Me di por vencido en creer en Perl como lenguaje de scripts orientado a objetos.
	\par
	Entonces me encontré con Python. Se trataba de un interprete y un lenguaje orientado a objetos. Pero no lo sentía como si fuera un "script" del lenguaje. Además, se trataba de un lenguaje híbrido de la programación de procedural y la programación orientada a objetos.
	\par
	Yo quería un lenguaje que fuera más poderoso que Perl, y más orientado a objetos que Python. Es por eso que me decidí a diseñar mi propio Lenguaje de programación''
\begin{flushright} Yukihiro Matsumoto (a.k.a. “Matz”)\end{flushright}
\end{quotation}  

Es así que comenzó a desarrollarlo el 24 de febrero de 1993 y en agosto desarrollo el primer Hello World con ruby. En 1994 se lanzó la primer versión alpha. Matz trabajó solo hasta el año 1996, momento en el cuál se formo la primer comunidad de Ruby.




% ¿QUE ES RUBY?
\section{¿Qué es Ruby?}

	Es un lenguaje escrito en C y fue diseñado teniendo en mente las capacidades de perl y phyton. [ Colocar contenido aquí ]




% EL LENGUAJE Y SUS PRINCIPIOS
\section{El lenguaje y sus Principios}
	
	Antes de iniciarnos en la sintaxis del lenguaje creemos necesario entender el por qué es que fue creado Ruby, es decir, entender el impacto que su creador deseaba que tuviera y las razones por las cuales conforma una forma muy particular de realizar aplicaciones.
	\par
	El lenguaje de programación Ruby es un ``\emph{lenguaje de programación interpretado para una rápida y fácil programación orientada a objetos}'' - Pero, ¿Qué significa esto? Veámos:
\bigskip\\

\textbf{Lenguaje de programación interpretado:}
\begin{itemize}
	\itemsep=1pt \topsep=0pt \partopsep=0pt \parskip=0pt \parsep=0pt
	\item habilidad para hacer llamadas directas al sistema operativo
	\item poderozo manejo y operaciones con cadenas y expresiones regulares
	\item respuestas inmediatas durante el desarrollo
\end{itemize}
\medskip

\textbf{Rápido y fácil:}
\begin{itemize}
\itemsep=2pt \topsep=0pt \partopsep=0pt \parskip=0pt \parsep=0pt
	\item innecesarias las declaraciones de variables
	\item variables no tipadas
	\item syntaxis simple y consistente
	\item el manejo de memoria es automático
\end{itemize}
\medskip

\textbf{Programación orientada a objetos:}
\begin{itemize}
\itemsep=2pt \topsep=0pt \partopsep=0pt \parskip=0pt \parsep=0pt
	\item todo es un objeto (como SmallTalk)
	\item clases, métodos, herencia, etc.
	\item métodos singleton (métodos que pertenecen a un sólo objeto)
	\item funcionalidad ``mixin'' por módulo
	\item iteradores y clausuras
\end{itemize}
\medskip

\textbf{Además:}
\begin{itemize}
\itemsep=2pt \topsep=0pt \partopsep=0pt \parskip=0pt \parsep=0pt
	\item números enteros de múltiple precisión
	\item procesamiento de excepciones
	\item carga dinámica
	\item soporte de concurrencia
\end{itemize}
\bigskip\bigskip

Sin duda, Ruby fue diseñado para hacer a los programadores mas productivos y felices. Esto es lo que el creador de Ruby nos dice:

\begin{quotation}
\em``Para mí, el propósito de la vida es, al menos en parte, tener alegría. Los programadores a menudo sienten alegría cuando se pueden concentrar en el aspecto creativo de la programación, por lo que Ruby está diseñado para que los programadores sean felices. Considero a un lenguaje de programación como una interfaz de usuario, por lo que deben seguir los principios de la interfaz de usuario.''

\begin{flushright} Yukihiro Matsumoto (a.k.a. “Matz”), 2000 \end{flushright}
\end{quotation}

\bigskip
¿Cuáles son los principios de una buena interfaz de usuario? Estos son los tres principios con citas de apoyo por parte de Matz:\\

\begin{quotation}
\noindent \textbf{Principio de concisión:} \textit{``Quiero que las computadoras sean mis siervos, no mis maestros. Por lo tanto, me gustaría darles órdenes rápidamente. Un buen siervo debe hacer un montón de trabajo con una breve orden.''}\\

\noindent \textbf{Principio de consistencia:} \textit{``... un pequeño conjunto de normas cubre la totalidad del lenguaje Ruby. Ruby es un lenguaje relativamente sencillo, pero no es demasiado simple. He tratado de seguir el principio de \textit{sin sorpresas}. Ruby no es demasiado único, por lo que un programador con conocimientos básicos de otros lenguajes de programación puede aprender muy rápidamente.''}\\

\noindent \textbf{Principio de flexibilidad:} \textit{``Porque las lenguas son para expresar el pensamiento, una lengua no debe restringir el pensamiento humano, sino que debe ayudarlo. Ruby consiste en un núcleo pequeño inmutable (es decir, sintaxis) y bibliotecas arbitrarias de clases extensibles...''}
\end{quotation}




% SINTAXIS
\section{Sintaxis}

	[ Colocar contenido aquí ]




% MANEJO DE MEMORIA
\section{Manejo de memoria}

	[ Colocar contenido aquí ]




% RECOLECCIÓN DE BASURA
\section{Recolección de basura}

	[ Colocar contenido aquí ]





% RUBY ON RAILS: UN PUNTO FUERTE DEL LENGUAJE
\section{\textit{Ruby On Rails}: un punto fuerte del lenguaje}

En la actualidad, Ruby se ha popularizado en el mundo del desarrollo de las aplicaciones webs a través del framework \textit{Ruby On Rails}, o mas comúnmente conocido como \textit{Rails}, escrito en este mismo lenguaje.\par
Ruby on Rails nace como un framework de desarrollo web especialmente diseñado con un fin en particular: hacer la vida más fácil a las personas que desarrollan aplicaciones destinadas a la web. Muchas de estas personas quiza se sientan frustradas con tecnologias como PHP, Java o .NET, que si bien son buenas, conllevan problemas de carga de recursos y una complejidad innecesaria. Ruby on Rails es simplemente más sencillo.\par
 Rails utiliza el patrón MVC para poder administrar sus recursos. Si bien Java utiliza distintos frameworks también basados en MVC, Rails lleva el concepto mucho mas allá debido a que hay un lugar específico para cada parte del codigo, y cada componente de nuestra aplicación funciona de manera estándar. Es decir, es como si iniciaramos una aplicacion con el esqueleto previamente armado.\par
Todas estas potenciales caracteristicas hicieron que Ruby on Rails sea una de las razones por las que el lenguaje de programación Ruby haya logrado un importante impulso, siendo cada vez mas reconocido como una buena opción a la hora de elegir un lenguaje con el que llevar a cabo un proyecto.




% APLICACIONES
\section{Aplicaciones}

Con el paso del tiempo Ruby fue utilizado por un número cada vez mayor de desarrolladores para llevar a cabo proyectos de grande envergadura. A continuación se muestra un listado de aplicaciones\footnote{''Véase una lista mas completa de aplicaciones en los siguientes vínculos: http://rubyonrails.org/applications\\http://www.ruby-lang.org/en/documentation/success-stories''} realizadas en este lenguaje:
\bigskip\\

\textbf{Web:}
\begin{itemize}
	\itemsep=1pt \topsep=0pt \partopsep=0pt \parskip=0pt \parsep=0pt
	\item Twitter (http://www.twitter.com)
	\item Shopify (http://www.shopify.com)
	\item Groupon (http://www.groupon.com)
	\item XING (http://www.xing.com)
\end{itemize}
\medskip

\textbf{Simulaciones:}
\begin{itemize}
	\itemsep=1pt \topsep=0pt \partopsep=0pt \parskip=0pt \parsep=0pt
	\item \textit{NASA Langley Research Center} usa Ruby para llevar a cabo simulaciones. (http://www.larc.nasa.gov)
	\item Motorola (http://www.motorola.com)
\end{itemize}
\medskip

\textbf{Modelado 3D:}
\begin{itemize}
	\itemsep=1pt \topsep=0pt \partopsep=0pt \parskip=0pt \parsep=0pt
	\item Google SketchUp (http://sketchup.google.com)
\end{itemize}
\medskip

\textbf{Telecomunicaciones:}
\begin{itemize}
	\itemsep=1pt \topsep=0pt \partopsep=0pt \parskip=0pt \parsep=0pt
	\item \textit{Open Domain Server}: permitir a los usuarios el uso de DNS dinámicos (http://ods.org)
	\item \textit{Lucent}: uso en producto de telefonía wireless 3G (http://www.lucent.com)
\end{itemize}
\medskip




% UN EJEMPLO PRACTICO DE APLICACIÓN
\section{Un ejemplo práctico de aplicacion}





% CONCLUSIÓN
\section{Conclusión}




\end{document}
